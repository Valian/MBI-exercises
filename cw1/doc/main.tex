\documentclass[a4paper]{article}
\usepackage[a4paper,  margin=1.0in]{geometry}

\usepackage{graphicx}
\usepackage{float}
\usepackage{hyperref}

\usepackage{polski}
\usepackage[utf8]{inputenc}
\begin{document}


\title{Ćwiczenie nr 1 z MBI, assembling DNA de-novo}
\author{Mateusz Chydziński, Michał Sypetkowski}
\maketitle

\section{Ogólne informacje}
Repozytorium git: \url{https://github.com/msypetkowski/MBI-exercises.git}


\section{Generowanie odczytów}
\begin{verbatim}
[pIRS] Bases in reference sequences:    5369772
[pIRS] Read pairs simulated:            1342443
[pIRS] Bases in reads:                  268488600
[pIRS] Coverage:                        50.00
[pIRS] Substitution error count:        3336118
[pIRS] Average substitution error rate: 1.243%
[pIRS] Insertion count:                 1144
[pIRS] Deletion count:                  2917
[pIRS] Average insertion rate:          0.00043%
[pIRS] Average deletion rate:           0.00109%
[pIRS] Average insertion length:        1.06
[pIRS] Average deletion length:         1.03
[pIRS] Fragments affected by GC bias:   65.71%
[pIRS] Bases masked by EAMSS algorithm: 0
\end{verbatim}

\section{Assembling de novo}

\begin{verbatim}
[2018-Apr-24 21:03:24.766157] [info] - num of sequences: 530
[2018-Apr-24 21:03:24.766199] [info] - sum: 12079573
[2018-Apr-24 21:03:24.766206] [info] - max: 708120
[2018-Apr-24 21:03:24.766217] [info] - average: 22791.646484
[2018-Apr-24 21:03:24.766224] [info] - median: 207.000000
[2018-Apr-24 21:03:24.766230] [info] - N50: 239670
\end{verbatim}


\subsection {Sprawdzenie wyników}

Użyliśmy wygenerowanego pliku report.tex :

\begin{table}[ht]
\begin{center}
\caption{All statistics are based on contigs of size $\geq$ 500 bp, unless otherwise noted (e.g., "\# contigs ($\geq$ 0 bp)" and "Total length ($\geq$ 0 bp)" include all contigs).}
\begin{tabular}{|l*{1}{|r}|}
\hline
Assembly & de\_novo\_output \\ \hline
\# contigs ($\geq$ 0 bp) & 530 \\ \hline
\# contigs ($\geq$ 1000 bp) & 113 \\ \hline
\# contigs ($\geq$ 5000 bp) & 100 \\ \hline
\# contigs ($\geq$ 10000 bp) & 93 \\ \hline
\# contigs ($\geq$ 25000 bp) & 76 \\ \hline
\# contigs ($\geq$ 50000 bp) & 56 \\ \hline
Total length ($\geq$ 0 bp) & 12079573 \\ \hline
Total length ($\geq$ 1000 bp) & 12001628 \\ \hline
Total length ($\geq$ 5000 bp) & 11965296 \\ \hline
Total length ($\geq$ 10000 bp) & 11907746 \\ \hline
Total length ($\geq$ 25000 bp) & 11627000 \\ \hline
Total length ($\geq$ 50000 bp) & 10894418 \\ \hline
\# contigs & 121 \\ \hline
Largest contig & 708120 \\ \hline
Total length & 12007762 \\ \hline
Reference length & 5369772 \\ \hline
GC (\%) & 67.31 \\ \hline
Reference GC (\%) & 67.32 \\ \hline
N50 & 239670 \\ \hline
NG50 & 437262 \\ \hline
N75 & 131596 \\ \hline
NG75 & 317536 \\ \hline
L50 & 17 \\ \hline
LG50 & 5 \\ \hline
L75 & 33 \\ \hline
LG75 & 9 \\ \hline
\# misassemblies & 7 \\ \hline
\# misassembled contigs & 7 \\ \hline
Misassembled contigs length & 2025014 \\ \hline
\# local misassemblies & 274 \\ \hline
\# unaligned mis. contigs & 0 \\ \hline
\# unaligned contigs & 0 + 11 part \\ \hline
Unaligned length & 171260 \\ \hline
Genome fraction (\%) & 99.191 \\ \hline
Duplication ratio & 2.222 \\ \hline
\# N's per 100 kbp & 2513.48 \\ \hline
\# mismatches per 100 kbp & 0.26 \\ \hline
\# indels per 100 kbp & 9.31 \\ \hline
Largest alignment & 691166 \\ \hline
Total aligned length & 11719348 \\ \hline
NA50 & 205870 \\ \hline
NGA50 & 418686 \\ \hline
NA75 & 118910 \\ \hline
NGA75 & 264348 \\ \hline
LA50 & 19 \\ \hline
LGA50 & 6 \\ \hline
LA75 & 37 \\ \hline
LGA75 & 11 \\ \hline
\end{tabular}
\end{center}
\end{table}

W doświadczeniu wykorzystany został genom bakterii Azorhizobium caulinodans ORS 571. Długość tego genomu wynosi 5369772 par nukleotydów, co oznacza że nie jest to złożony orgaznim. W pierwszym kroku zostały wygenerowane odczyty sekwencji genomu za pomocą narzędzia pirs. W kolejnym etapie narzędzie dnaasm dokonało assemblingu "de novo" sekwencji, bazując na symulowanych odczytach. Jakość assemblingu obrazuje powyższy raport uzyskany z narzędzia quast. Uzyskano w sumie 530 kontigów, z czego znaczna większość (417) jest długości mniejszej niż 1000 bp. Najmniej kontigów mieści się w przedziale (1000 bp, 25000 bp) - jest ich jedynie 7. Najdłuższy wskazany kontig ma długość 708120 bp. Zawartość procentowa kwasów nukleinowych G-C wynosi 67.31,co różni się jedynie o 0.01 punktu procentowego od wartości referencyjnej. Świadczy to o dobrej jakości assemblingu, która zachowała stosunek kwasów G-C do A-T w oryginalnym genomie. 

\end{document}
