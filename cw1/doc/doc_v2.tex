\documentclass[a4paper]{article}
\usepackage[a4paper,  margin=0.7in]{geometry}

\usepackage{graphicx}
\usepackage{float}
\usepackage{hyperref}

\usepackage{polski}
\usepackage[utf8]{inputenc}
\begin{document}


\title{Ćwiczenie nr 1 z MBI, assembling DNA de-novo}
\author{Kinga Kimnes, Jakub Skałecki}
\maketitle

\section{Ogólne informacje}

Do przeprowadzenia doświadczenia wykorzystano genom referencyjny bakterii śluzowej Chondromyces crocatus, którego długość wyniosła 11388132 bp.


\section{Generowanie odczytów}

Przy użyciu aplikacji pIRS(Profile-based Illumina pair-end Reads Simulator) wygenerowano zestaw odczytów sekwencji genomu. Poniżej zamieszczono fragment logów aplikacji:

\begin{verbatim}
[pIRS] Bases in reference sequences:    11388132
[pIRS] Read pairs simulated:            2847033
[pIRS] Bases in reads:                  569406600
[pIRS] Coverage:                        50.00
[pIRS] Substitution error count:        7052256
[pIRS] Average substitution error rate: 1.239%
[pIRS] Insertion count:                 2525
[pIRS] Deletion count:                  6056
[pIRS] Average insertion rate:          0.00044%
[pIRS] Average deletion rate:           0.00106%
[pIRS] Average insertion length:        1.07
[pIRS] Average deletion length:         1.03
[pIRS] Fragments affected by GC bias:   69.08%
[pIRS] Bases masked by EAMSS algorithm: 0
\end{verbatim}

\section{Assembling de novo}

W dalszej kolejności, na podstawie zasymulowanych odczytów, dokonano asemblacji "de novo" sekwencji  z wykorzystaniem narzędzia dnaasm. Jej ocenę przeprowadzono przy użyciu narzędzia QUAST (uzyskany raport - tabela \ref{fig:table1}).

W sumie uzyskano 3185 kontigów, z których jedynie 422 posiadają długość większą niż 500bp, a 155 długość większą niż 50000bp. Najdłuższy kontig liczy 377722bp. Zawartość nukleotydów G i C w uzyskanej sekwencji wynosi 68,66\% (GC), co odbiega od wartości referencyjnej jedynie o 0,05\% (Reference GC = 68,71\%) i stanowi spełnienie istotnego warunku dobrej jakości asemblacji. Całkowita długość (łączna liczba par nukleotydów) uzyskanego łańcucha stanowi dwukrotność długości genomu referencyjnego, przy czym genom ten został odtworzony w 98,32\% (Genome fraction). Liczba źle zasemblowanych kontigów wynosi 10, a ich łączna długość stanowi około 3,6\% całkowitej długości wszystkich kontigów.

Część logów z assemblingu:
\begin{verbatim}
[2019-Nov-10 13:04:37.209073] [info] - num of sequences: 3185
[2019-Nov-10 13:04:37.209112] [info] - sum: 23330999
[2019-Nov-10 13:04:37.209121] [info] - max: 377722
[2019-Nov-10 13:04:37.209131] [info] - average: 7325.274902
[2019-Nov-10 13:04:37.209138] [info] - median: 148.000000
[2019-Nov-10 13:04:37.209145] [info] - N50: 118698
\end{verbatim}

\newpage
\begin{table}[ht]
\section{Zawartość pliku report.txt:}
\begin{center}
\caption{Wszelkie statystyki oparte są o kontigi o rozmiarze $\geq$ 500 bp, jeśli nie zanotowano inaczej (np. parametry "\# contigs ($\geq$ 0 bp)" lub "Total length ($\geq$ 0 bp)" zawierają wszystkie kontigi).
\label{fig:table1}
}
\begin{tabular}{|l*{1}{|r}|}
\hline
Assembly & de\_novo\_output \\ \hline
\# contigs ($\geq$ 0 bp) & 3185 \\ \hline
\# contigs ($\geq$ 1000 bp) & 388 \\ \hline
\# contigs ($\geq$ 5000 bp) & 304 \\ \hline
\# contigs ($\geq$ 10000 bp) & 258 \\ \hline
\# contigs ($\geq$ 25000 bp) &  216\\ \hline
\# contigs ($\geq$ 50000 bp) & 155 \\ \hline
Total length ($\geq$ 0 bp) & 23330999 \\ \hline
Total length ($\geq$ 1000 bp) & 22891314  \\ \hline
Total length ($\geq$ 5000 bp) & 22677669 \\ \hline
Total length ($\geq$ 10000 bp) & 22353298 \\ \hline
Total length ($\geq$ 25000 bp) & 21697226 \\ \hline
Total length ($\geq$ 50000 bp) & 19504676\\ \hline
\# contigs & 422 \\ \hline
Largest contig & 377722 \\ \hline
Total length  & 22915670 \\ \hline
Reference length & 11388132 \\ \hline
GC (\%) & 68.66 \\ \hline
Reference GC (\%) & 68.71 \\ \hline
N50 & 118995  \\ \hline
NG50 & 210841 \\ \hline
N75  & 72352  \\ \hline
NG75 & 169742 \\ \hline
L50 & 57 \\ \hline
LG50 & 21 \\ \hline
L75 & 116 \\ \hline
LG75 & 37 \\ \hline
\# misassemblies & 10 \\ \hline
\# misassembled contigs & 10 \\ \hline
Misassembled contigs length & 821997 \\ \hline
\# local misassemblies & 705 \\ \hline
\# unaligned mis. contigs & 1 \\ \hline
\# unaligned contigs & 0 + 61 part \\ \hline
Unaligned length & 225334 \\ \hline
Genome fraction (\%) & 98.318 \\ \hline
Duplication ratio & 2.027  \\ \hline
\# N's per 100 kbp & 2263.76  \\ \hline
\# mismatches per 100 kbp & 3.05 \\ \hline
\# indels per 100 kbp & 12.00 \\ \hline
Largest alignment & 369436 \\ \hline
Total aligned length & 22421453 \\ \hline
NA50 & 114050  \\ \hline
NGA50 & 201381 \\ \hline
NA75 & 65918 \\ \hline
NGA75 & 158761 \\ \hline
LA50 & 61 \\ \hline
LGA50 & 22 \\ \hline
LA75 & 123 \\ \hline
LGA75 & 38 \\ \hline
\end{tabular}
\end{center}
\end{table}

\end{document}