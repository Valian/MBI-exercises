\documentclass[a4paper]{article}
\usepackage[a4paper,  margin=1.0in]{geometry}

\usepackage{graphicx}
\usepackage{float}
\usepackage{hyperref}
\usepackage{listings}
\lstset{
basicstyle=\small\ttfamily,
columns=flexible,
breaklines=true
}

\usepackage{polski}
\usepackage[utf8]{inputenc}
\begin{document}


\title{Ćwiczenie nr 3 z MBI, Resekwencjonowanie genomu człowieka}
\author{Mateusz Chydziński, Michał Sypetkowski}
\maketitle

\section{Ogólne informacje}
Repozytorium git: \url{https://github.com/msypetkowski/MBI-exercises.git}.
W katalogu \texttt{cw3} repozytorim zawiera skrypty/polecenia użyte do przeprowadzenia eksperymentów.

\section{Mapowanie}

Najpierw indeksujemy plik FASTA (w pliku jest jedna sekwencja) genomu referencyjnego.

Następnie generujemy plik SAM (Sequence Alignment Map) za pomocą algorytmu bwa mem.
Plik ten zawiera zbiór mapowań sekwencji na (ang. aligned to) do sekwencji referencyjnej.
W naszym przypadku sekwencja referencyjna to plik chr1.fa,
a na tą sekwencę mapujemy sekwencje (odczyty z sekwencera)
z pliku coriell\_chr1.fq (zawiera on około 284k sekwencji).
Plik SAM zawiera mapowania kolejnych sekwencji z pliku coriell\_chr1.fg.

Następnie sortujemy te mapowania (mapowanie zawiera przy okazji sekwencje)
oraz tworzymy ich binarną reprezentację (plik BAM).
Stąd plik SAM ma 73MB, a BAM 14MB (mniej).
Plik FASTQ ze zbiorem sekwencji zawiera 55M --
mniej niż SAM, bo nie informacji o mapowaniu na genom referencyjny.
Plik FASTQ zawiera sekwencje (odczyty) i "quality value characters" dla każdej sekwencji równe jej długości.
Plik SAM również zawiera sekwencje (tyle samo i takiej samej długości co FASTQ) i analogicznie "quality value characters".

Średnia długość sekwencji odczytu to 75.5 z odchyleniem standardowym 1.17.

\end{document}
