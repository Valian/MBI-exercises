\documentclass[a4paper]{article}
\usepackage[a4paper,  margin=1.0in]{geometry}

\usepackage{graphicx}
\usepackage{float}
\usepackage{hyperref}
\usepackage{listings}
\lstset{
basicstyle=\small\ttfamily,
columns=flexible,
breaklines=true
}

\usepackage{polski}
\usepackage[utf8]{inputenc}
\begin{document}


\title{Ćwiczenie nr 4 z MBI, Analiza danych sekwencyjnych człowieka}
\author{Mateusz Chydziński, Michał Sypetkowski}
\maketitle

\section{Ogólne informacje}
Repozytorium git: \url{https://github.com/msypetkowski/MBI-exercises.git}.
W katalogu \texttt{cw4} repozytorim zawiera skrypty/polecenia użyte do przeprowadzenia eksperymentów.


\section{Liczenie pokrycia}
Największa mediana pokrycia została zarejestrowana dla próbki \texttt{NA19137} i wynosi \texttt{76.03475}. Najmniejsza wyliczona
mediana charakteryzuje próbkę o numerze \texttt{NA18991} a jej wartość to \texttt{21.06271}.


\section{Wykrywanie zmian liczby kopii DNA przy użyciu narzędzia CODEX}


\section{Wnioski}


\end{document}
